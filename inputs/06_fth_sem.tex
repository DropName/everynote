Оценим теперь для последней задачи характерную длину волны излучения.
Работали в предположении $v \ll c$, и у нас $\vc{\ddot{d}} \parallel \vc{\ddot{r}}$.

Чтобы не тратить целую пару на вывод, а предоставить работу лектору, сразу напишем формулу:
\begin{equation*}
	1 - \frac{v}{c} \cos \theta = z (\theta),
\end{equation*}
где $z (\theta=0)$  будем минимумом. Далее
\begin{equation*}
	z(\theta) = 2 z (0)
	\hspace{1 cm}
	\Rightarrow
	\hspace{1 cm}
	1 - \frac{v}{c} \cos \theta = 2 \left(1 - \frac{v}{c}\right).
\end{equation*}

Если же теперь $v/c \to 1$, тогда $\theta \to 0$, и имеем в приближении наше уравнение:
\begin{equation*}
	1 - \frac{v}{c} + \frac{v}{c} \frac{\theta^2}{2} = 2 - 2 \frac{v}{c}
	\hspace{1 cm}
	\Rightarrow
	\hspace{1 cm}
	\theta^2 = (1 - \frac{v^2}{c^2}) = \gamma^{-2}.
\end{equation*}
Для оценки получили, что $\theta \sim \gamma^{-1}$. Время которое частица излучает: $\Delta t_{\text{изл}} = \frac{\theta}{2\pi} T$.

То есть если $t_1$ -- время, когда диполь начал излучать, а $t_2$ -- время, когда закончил, то длина $l = (t_2 - t_1)v$, а длина фотона $l_{\text{фот}} = (t_2 - t_1)c$.
Тогда \begin{equation*}
	\Delta = l_{\text{фот}} - l = (c-v)(t_2 - t_1) = (c-v) \Delta t_{\text{изл}}.
\end{equation*}
А время, которое детектирует детектор $\Delta t_\text{дет} = \Delta l/c$.
Таким образом:
\begin{equation*}
	\Delta t_{\text{дет}} = \left(1 - \frac{v}{c}\right) \Delta t_{\text{изл}} \simeq  \frac{1}{2} \gamma^{-2} \Delta t_{\text{изл}}.
\end{equation*}
И в итоге $\Delta t_{\text{дет}} \sim \frac{\theta}{2 \pi} \gamma^{-2} T \sim \gamma^{-3} T$.

Из фурье анализа $\omega \Delta t_\text{дет} \simeq 2 \pi$. Тогда $\omega \simeq \gamma^3 \Omega$, где $\Omega$ -- циклотронная частота. Которая в свою очередь где-то примерно порядка $\omega \sim 10^{20}\ \text{c}^{-1}$.

\subsubsection*{Задача 20}
Имеем что? $\vc{E}_0 \parallel \vc{e}_x$ и $\vc{H}_0 \parallel \vc{e}_y$. Запишем вектор Пойтинга для такой рассейянной волны:
\begin{equation*}
	\vc{S} = \frac{c}{4 \pi} |H|^2 \vc{n} = \frac{c}{4 \pi} |\vc{E}|^2 \vc{n}.
\end{equation*}
Вдали от источника, как мы обсуждали выполняется $\vc{E} \perp \vc{H}$ и равны по модулю.

Для интенсивности имеем
\begin{equation*}
	d I = \vc{n} \vc{S} r^2 d \Omega = S_0 d \sigma,
\end{equation*}
где ввели дифференциальное сечение рассеяния $d \sigma$, а $S_0 = \frac{c}{4 \pi} |\vc{E}_0|^2$.

Внутри идеально проводящего шара $\vc{E} = \vc{0}$ и $\vc{H} = \vc{0}$.
Рассмотрим конкретно электрическое поле в центре шара с плотностью заряда $\rho(\theta,\phi)$, взяв закон Кулона:
\begin{equation*}
	\vc{E}_0 \cos (\omega t - \vc{k} \cdot \vc{r}) + \int \frac{\rho (\theta,\varphi) (-  \vc{r})}{r^3}d S = 0
	\hspace{1 cm}
	\Rightarrow
	\hspace{1 cm}
	\vc{E}_0 \cos (\omega t) - \frac{1}{r^3} \underbrace{\int \rho(\theta,\varphi) \vc{r} d S}_{\vc{d}} = 0.
\end{equation*}
Соответственно получаем: $\vc{d} = \vc{E}_0 r^3 \cos (\omega t)$.

Теперь берем Био-Савара
\begin{equation*}
	\vc{H}_0 \cos (\omega t) + \int \frac{[\vc{J} \times (- \vc{r})]}{c r^3} dS = 0
	\hspace{1 cm}
	\Rightarrow
	\hspace{1 cm}
	\vc{H}_0 \cos (\omega t) + \frac{2}{r^3} \int \frac{\vc{r} \times \vc{J}}{2 c} d S = 0.
\end{equation*}
И аналогично $\vc{\mu} = - \frac{\vc{H}_0 r^3}{2} \cos (\omega t)$.

В волновой (зоне) будет верно, что
\begin{equation*}
	\vc{H}_d = \frac{\vc{\ddot{d}} \times \vc{n}}{c^2 r},
	\hspace{1 cm}
	\vc{H}_\mu = \frac{\vc{n} (\vc{n} \cdot \vc{\ddot{\mu}}) - \vc{\ddot{\mu}}}{c^2 r}.
\end{equation*}
Таким образом вектор Пойтинга:
\begin{equation*}
	\vc{S} = \frac{c}{4 \pi} |\vc{H}_d + \vc{H}_\mu|^2 \vc{n}
	=
	\frac{c}{4 \pi c^4 r^2} \big( |[\vc{\ddot{d}} \times \vc{n}]|^2 + (\vc{n} \cdot \vc{\ddot{\mu}})^2 + |\vc{\ddot{\mu}}|^2 + 2 ([\vc{\ddot{d}} \times \vc{n}] \cdot \vc{n}) (\vc{n} \cdot \vc{\ddot{\mu}}) - 2 (\vc{\ddot{\mu}} \cdot [\vc{\ddot{d}} \times \vc{n}]) - 2 (\vc{n} \cdot \vc{\ddot{\mu}})^2 \big) \vc{n}.
\end{equation*}

Будем разбираться по очереди: $[\vc{\ddot{d}} \times \vc{n}] \cdot \vc{n} = 0$.

Далее: $[\vc{\ddot{d}} \times \vc{n}]_\alpha [\vc{\ddot{d}} \times \vc{n}]^\alpha = |\vc{\ddot{d}}|^2 - (\vc{n} \cdot \vc{\ddot{d}})^2$.

Теперь вроде как немного упростилось:
\begin{equation*}
	\vc{S} = \frac{1}{4 \pi c^3 r^2} \big( |[\vc{\ddot{d}} \times \vc{n}]|^2 - (\vc{n} \cdot \vc{\ddot{\mu}})^2 + |\vc{\ddot{\mu}}|^2 - (\vc{n} \cdot \vc{\ddot{\mu}})^2 - 2 (\vc{\ddot{\mu}} \cdot [\vc{\ddot{d}} \times \vc{n}]) \big) \vc{n}.
\end{equation*}
И теперь по формулам выше найдём сечение:
\begin{equation*}
	d \sigma = \frac{\omega^4 r^6}{c^4} \cos^2 (\omega t)\big( |\vc{e}_x|^2 - (\vc{n} \cdot \vc{e}_x)^2 + \frac{1}{4}|\vc{e}_y|^2 - \frac{1}{4} (\vc{n} \cdot \vc{e}_y)^2 - (\vc{e}_y [\vc{e}_x \times \vc{n}])\big) d \Omega 
	= 
	\boxed{
	\frac{\omega^4 r^6}{2 c^4}
	\big( \frac{5}{4} - \sin^2 \theta \cos^2 \varphi - \frac{1}{4} \sin^2 \theta \sin^2 \varphi + \cos \theta\big) d \Omega}
\end{equation*}
А теперь, как нас просят  задаче, мы это возьмём и проинтегрируем!
\begin{equation*}
	\sigma = 2 \pi \frac{\omega^4 r^6}{2 c^4} (\frac{5}{4} -\frac{1}{3} - \frac{1}{12}) =  \frac{5 \pi}{3} \frac{\omega^4}{2 c^4} r^6.
\end{equation*}