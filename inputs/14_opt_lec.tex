 \subsection*{Оптика пучков}
 Что у нас есть. У нас есть волновое уравнение!
 \begin{equation*}
 	\nabla^2 E - \frac{\varepsilon \mu}{c^2} \frac{\partial^2 E}{\partial t^2} = 0.
 \end{equation*}
 Получаем уравнение Гельмгольца:
 \begin{equation*}
 	\nabla^2 f e^{- i \omega t} - \frac{\varepsilon \mu}{c^2} (-1) \omega^2 f e^{-i \omega t} = 0
 	\hspace{0.5 cm}
 	\Rightarrow
 	\hspace{0.5 cm}
 	\nabla^2 f + \varepsilon \frac{\omega^2}{c^2} f = 0.
 \end{equation*}
 Таким образом:
 \begin{equation}
 	\nabla^2 f + \varepsilon k_0^2 f = 0
 	\label{helmgoltz}
 \end{equation}

 Если рассматривать полну идущую от точечного источника, то вблизи мы будет ее считать \textit{сферической} чуть дальше --- \textit{параболической}, и совсем далеко --- \textit{плоской}.  

 Выберем ось $z$ на которой поместим точечный источник.
Теперь $f = \frac{A}{r} \exp(i k r)$.

На расстоянии $\rho$ от оси точка на волновом фронте лежит на $r^2 = \rho^2 + z^2$. Приблизительно в предположении $z \gg \rho$: $r \approx z + \frac{\rho^2}{2 z}$.
То есть в порядке малости выпишем:
\begin{equation*}
	\encircled{1} \, r \simeq z,
	\hspace{1 cm}
	\encircled{2} \, r \simeq \frac{2 \pi}{\lambda} (z + \frac{\rho^2}{2 z} + \ldots) = \frac{2\pi}{\lambda}z + \frac{2 \pi}{\lambda} \frac{\rho^2}{2 z}.
\end{equation*}
Второй вот порядок и называют параболическим: $f = \frac{A}{z} \exp(i k z + i k \frac{\rho^2}{2 z})$.

\subsection*{Параксиальное приближение}
Имеем ситуацию: $f (\vc{r}) = A (\vc{r}) \exp(i k z)$. Будем требовать:
\begin{equation*}
	\frac{\partial A}{\partial z}\lambda \ll A 
	\hspace{0.3 cm}
	\Rightarrow
	\hspace{0.3 cm}
	\frac{\partial A}{\partial z} \ll \frac{A}{\lambda}
	\hspace{0.3 cm}
	\Rightarrow
	\hspace{0.3 cm}
	\frac{\partial A}{\partial z} \ll A \cdot k.
\end{equation*}
\begin{equation*}
	\frac{\partial^2 A}{\partial z^2}\lambda \ll \frac{\partial A}{\partial z} 
	\hspace{0.3 cm}
	\Rightarrow
	\hspace{0.3 cm}
	\frac{\partial^2 A}{\partial z^2} \ll \frac{\partial A}{\partial z} \cdot k.
\end{equation*}
И так волновое уравнение, с заменой $k^2 = \varepsilon \omega^2/c^2$:
\begin{equation*}
	\nabla^2 f + k^2 f  = 0,
	\hspace{0.5 cm}
	\hspace{0.5 cm}
	\frac{\partial f}{\partial z} = \frac{\partial A}{\partial z} e^{i k z} + A \cdot i k e^{i k z},
	\hspace{0.5 cm}
	\hspace{0.5 cm}
	\frac{\partial^2 f}{\partial z^2} = \frac{\partial^2 A}{\partial z^2} e^{i k z} + 2 A_z' i k e^{i k z} - A k^2 e^{i k z}.
\end{equation*}
Теперь всё это в волновое уравнение:
\begin{equation*}
	\frac{\partial^2 A}{\partial z^2} + 2 \frac{\partial A}{\partial z} i k - A k^2 + \nabla^2 A + A k^2 = 0
	\hspace{0.5 cm}
	\Rightarrow
	\hspace{0.5 cm}
	\frac{\partial^2 A}{\partial z^2} + 2 \frac{\partial A}{\partial z} i k + \nabla^2 A  = 0.
\end{equation*}
И с нашем приближение получаем \textit{параксиальное уравнение Гельмгольца} (может мы где-то знак потеряли).
\begin{equation}
 	\nabla^2 A + 2 i k \frac{\partial A}{\partial z} = 0.
 	\label{parax_helm}
 \end{equation}
 Надо проверить, подставится будет ли решением уравнения сверху выражение:
 \begin{equation*}
 	f (r) = \frac{A}{z} \exp\left(-i k z - i k \frac{\rho^2}{2 z}\right)
 	\hspace{0.5 cm}
 	\Rightarrow
 	\hspace{0.5 cm}
 	f(r) = A(r) e^{- i k z},
 	\hspace{0.3 cm}
 	\hspace{0.3 cm}
 	A(r) = \frac{A}{z} \exp\left(-i k \frac{\rho^2}{2 z}\right).
 \end{equation*}
 Решение таким образом при таким $f$.	
 \begin{equation*}
 	f (r) = \frac{A}{z + i z_0} \exp\left(-i k z - i k \frac{\rho^2}{2 (z + i z_0)}\right)
 	\hspace{0.5 cm}
 	\hspace{0.5 cm}
 	z \longrightarrow q(z) = z + i z_0.
 \end{equation*}
 Тут введен $q$-параметр и \textit{Рэлеевская длина} --- $z_0$.

 Мы знаем, что  $\frac{1}{q(z)} = \frac{1}{z + i z_0} = \frac{z - i z_0}{z^2 + z_0^2}$. Подставляем:
 \begin{equation*}
 	f (r) = A \left(\frac{z}{z^2 + z_0^2} - i\frac{z_0}{z^2 + z_0^2}\right) \exp\left(- i k z - i k \frac{\rho^2}{2}\frac{z - i z_0}{z^2 + z_0^2}\right).
 \end{equation*}
 \begin{equation*}
 	f (r) = A \left(\frac{z}{z^2 + z_0^2} - i\frac{z_0}{z^2 + z_0^2}\right)
 	\exp \left( - \frac{k \rho^2}{2} \frac{z_0}{z^2 + z_0^2}\right)
 	\exp\left(- i k z - i k \frac{\rho^2 z_0}{z^2 + z_0^2}\right).
 \end{equation*}
Выпишем:	
\begin{equation*}
	- \frac{2 \pi}{2} \frac{ \rho^2 z_0}{\lambda (z^2 + z_0^2)} = i \frac{\rho^2}{\frac{\lambda}{z_0 \pi} (z^2 + z_0^2)}
	\hspace{1 cm}
	\text{обозначим: }
	W^2(z) = \frac{\lambda z_0}{\pi} (1 + \frac{z^2}{z_0^2}),
	\hspace{0.5 cm}
	R(z) = z(1 + \frac{z^2}{z_0^2}).
\end{equation*}
Тогда получаем:
\begin{equation*}
	f(r) = A \left(\frac{1}{R(z)} - i \frac{\lambda}{\pi W^2(z)}\right) \exp(- \frac{\rho^2}{W^2(z)}) \exp\left(- i k z - i k \frac{\rho^2}{2 R(z)}\right).
\end{equation*}
\begin{equation*}
	f(r) = \frac{A}{i z} \frac{W_0}{W(z)} \exp\left(-\frac{\rho^2}{W^2(z)}\right) 
	\exp\left(- i k z - i k \frac{\rho^2}{2 R(z)} + i \xi(z) \right),
\end{equation*}
где $W_0 = W(0)$, а $\xi(z) = \arctan\frac{z}{z_0}$. Обычно в первом члене ещё обозначают $A_0 = A/i z_0$.

\subsubsection*{Интенсивность}
\begin{equation*}
	I = <|\vc{S}|>_t = <\frac{c}{4 \pi} \sqrt{\varepsilon} E^2> = \frac{c n}{4 \pi}<E^2> = \frac{c n}{8 \pi} E_0^2.
\end{equation*}
И для всех будет прекрасней, для всех будет полезней не таскать коэффициент:
\begin{equation*}
	I \propto E_0^2 \hspace{0.5 cm} I \propto E^2 \hspace{0.5 cm} 
	I = E^2 \left[\frac{\text{B}^2}{\text{м}^2}\right],
\end{equation*}
сокращать на размерный коэффициент очень круто, поэтому запомним, что размерность $[I] = \frac{\text{Дж}}{\text{c cm}}$.

Таким образом со всеми предположениями:
\begin{equation*}
	I = A_0^2 \frac{W_0^2}{W^2(z)} \exp\left(- \frac{2 \rho^2}{W^2(z)}\right).
\end{equation*}
Это называется \textit{Гаусовой интенсивностью}. Читатель может построить её самостоятельно.

\subsubsection*{Мощность}
\begin{equation*}
	P = \int_0^\infty I(\rho) 2 \pi \rho d \rho = \frac{1}{2} I_0 (\pi W_0^2).
\end{equation*}
И в построенном читателями графике в предыдущем пункте если посмотреть на ширину полосы под распределением, то можно заметить, что $\rho_0 = W(z)$, в связи с чем $W(z)$ называют \textit{радиусом пучка}.

Посмотрим, что у нас с нашим радиусом творится:
\begin{equation*}
	W(z) = \sqrt{\frac{\lambda z_0}{\pi} \left(1 + \frac{z^2}{z_0^2}\right)},
\end{equation*}
если пытливый читатель построить и этот график, то он (график) пересечет $z =0$ в $W(0) = W_0 = \sqrt{\frac{\lambda z_0}{\pi}}$. Aсимптотически он будет стремиться к прямой с коэффициентом наклона: $\theta \approx \sqrt{\frac{\lambda}{\pi z_0}} = \frac{W_0}{z_0}$.

Для примера если взять лазер $\lambda_0 = 633$ нм, с $2W_0 = 2$ см, то \textit{глубина резкости} будет: $2 z_0 = 1$ км. Если же $2W_0 = 2$ мкм, то уже всё будет потоньше: $2z_0 = 1$ нм.