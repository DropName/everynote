\subsubsection*{Про охлаждение лазерами}

Для нас будет очень важна фазовая плотность. 
В осях $px$ на элементарный объем приходится ровно $2 \pi \hbar$.
\begin{equation*}
	N = \int \frac{d^3 x \d^3 p}{(2 \pi \hbar)^3}\frac{N}{V}\left(\frac{2 \pi \hbar^2}{m T}\right)^{3/2} e^{- p^2/2 mT}.
\end{equation*}
Таким образом фазовая плотность примерно:
\begin{equation*}
	~ \frac{N}{V}\left(\frac{2 \pi \hbar^2}{m T}\right)^{3/2} \equiv \frac{N}{V} \lambda_{d B}^3
\end{equation*}
Для начала возьмем нормальное распределение:
\begin{equation*}
	N(v) = v^2 e^{-m v^2/T}.
\end{equation*}
И если наиболее вероятной скорости соответствовало что-то вроде $v =\sqrt{T/m}$. И нача цель сдвинуть этот пик как можно ближе к нулю.

\begin{to_ht}
	Оценить:
	\begin{itemize}
	\item фазовую плотность азота в атмосфере при нормальных температурах
	\item фазовую плотность атомов лития при температуре $100$ мкК и межчастичном расстоянии $1$ мкм.
\end{itemize}	
\end{to_ht}


Простейший вариант описывающий охлаждения (замедление) атома светом, это посветить светом навстречу летящему атому, чтобы уменьшить его скорость.
Будем смотреть на модель двухуровнего атома, который возбуждается на частоте $\omega_{laser}$. Атом кушает фотон, и переходит с первого на второй энергетический уровень.

Ускорение, которое приобретает атом при давлении на него света:
\begin{equation*}
	m a = \hbar k \frac{\Gamma}{2},
\end{equation*}
где $\hbar k$ -- импульс одного возбужденного фотона, а $\Gamma$ -- обратное время жизни возбужденного состояния атома.

\begin{to_ht}
	Атом лития замедляется резонансным светом при переходе $2s \to 2 p$. Рассчитайте ускорение и сравните с $g$. Замедление начинается со скорости $500$ м/с. Рассчитайте расстояние на котором скорость падает почти до нуля.
\end{to_ht}

Для удержания вещества используется \textbf{магнито-оптическая ловушка}.
Будем пользоваться \textit{эффектом Зеймана}. Также потребуем от падающего с разных сторон света обратной поляризации $\sigma_-$ и $\sigma_+$.

А какие у нас вообще пределы всего этого:
\begin{equation*}
	T_{\text{min}} = \frac{p_{\text{photon}}^2}{2 m}, \ p_{\text{photon}} = \frac{2 \pi \hbar}{\lambda}
	\hspace{0.5 cm}
	\leadsto
	\hspace{0.5 cm}
	T_{\text{min}} = 4 \text{ мкК (для Li)}.
\end{equation*}

И ещё есть так называемый \textit{предел Летохова-Миногина-Павлика}:
\begin{equation*}
	T_{\text{min}} = \frac{\hbar \Gamma}{2} = 150 \text{ мкК}.
\end{equation*}

Чтобы пойти ещё холоднее воспользуемся простой \textbf{оптической дипольной ловушкой}.
Нас нужно, чтобы собственная частота атома была много больше частоты колебаний электрического поля, чтобы можно было считать поле от пришедшего света в принципе постоянным.
\begin{equation*}
	\varphi = - \vc{d} \cdot \vc{E} = -\frac{\alpha}{2} E^2.
\end{equation*}
Тот же принцип применяется в оптическом пинцете.

И вот мы удержали, теперь будем охлаждать испарением. А именно, те атомы, которые движутся достаточно быстро, чтобы вылететь из ямы ее и покидают, остается меньше, оставшиеся вновь пытаются выстроится нормально и снова появляется кто-то быстрый, кто снова вылетит. Таким образом наблюдаем что-то похожее на испарение.

\begin{to_ht}
	Посмотрим на сложность такого метода и подводные камни (Li):
	\begin{equation*}
		\nu = n \sigma v
	\end{equation*}
	и тут можно уже увидеть интересный результат для сечения столкновения и частоты столкновения, которые вообще для нормального испарения мы хотим получать большими.

	И то же самое проделать для частиц в атмосфере.
\end{to_ht}

\subsubsection*{Квантовые вычисления на холодных ионах}
Хотим записать гамильтониан иона в ловушке и изучить их движение и взаимодействие ионов в ловушке с внешним полем.

Квантовые механические системы описываются уравнением шредингера:
\begin{equation*}
	i \hbar \partial_t \Psi = \hat{H} \Psi,
\end{equation*}
где $\Psi(r_1,\ldots,r_N, t)$ -- плотность вероятности нахождения $N $ частиц в области пространства.
И гамильтониан соотвественно:
\begin{equation*}
	\hat{H} = - \sum_i \frac{\hbar}{2 m} \frac{\partial^2}{\partial x_i^2} + V(x_1,\ldots,x_n),
\end{equation*}
где $V$ -- потенциальна энергия системы частиц, которая задается как:
\begin{equation*}
	V = \sum_{i,j} \frac{e^2}{|r_i - r_j|} + \sum V(r_i).
\end{equation*}
И у уравнения есть стационарное решение:
\begin{equation*}
	\Psi = e^{- \frac{i E_n t}{\hbar}} \Psi(r_1, \ldots, r_n).
\end{equation*}

Это все случилось для одного иона, у которого есть $N$ электронов, теперь поместим это все во внешнее электрическое поле.
\begin{equation*}
	\hat{V} = - e \vc{E}(t) \cdot (\vc{r}_1+ \ldots+ \vc{r}_n) = \vc{E} \cdot \vc{d}.
\end{equation*}
Допустим мы нашли все уровни энергии, такие что $\bar{H}_0 \Psi_n = E_n \Psi_n$, для нашего гамильтониана вида $H = \bar{H}_0 + \bar{V}$. Тогда решения для уравнения Шредингера будет выглядеть как
\begin{equation*}
	\Psi = \sum c_n (t) e^{-i E_n t} \Psi_n
\end{equation*}
И само уравнения запишется:
\begin{equation*}
	i \hbar \partial_t c_n = \sum_{m} e^{- (E_n - E_m)t} V_{n m} c_m.
\end{equation*}
Из всего такого набора $c_n$ оставим только резонансные. И то, что мы светим в резонанс означает, что $\vc{E}(t) = E_0 \cos \left((E_1 - E_0) t\right)$.

Мы ограничимся всего двумя уровнями энергиями, $c_0$ и $c_1$:
\begin{equation*}
	i \partial_t \begin{pmatrix} c_0 \\ c_1 \end{pmatrix} = \begin{pmatrix}
	    0 & (E d)^* \\
	    (E d) & 0 \\
	\end{pmatrix} \begin{pmatrix} c_0 \\ c_1 \end{pmatrix}.
\end{equation*}
В результате получаются просто осциллируют:
\begin{equation*}
	c_0(t) = c_0 (0) \cos \left(\frac{|E_0 d|}{\hbar} t\right) + i \sin \left(\frac{|E_1 d|}{\hbar}t\right) e^{i \varphi} c_1 (0)
\end{equation*}
Вспомнив про матрицы Паули:
\begin{equation*}
	\sigma_{+,-} = \frac{\sigma_x \pm i \sigma_y}{2}.
\end{equation*}
И далее в более удачном виде все у нас запишется...
\begin{equation*}
	i \hbar \partial_t c = (E d)^* c \sigma_- + (E d) c \sigma_+.
\end{equation*}



Теперь от электронной системы перейдем к системе в которой ещё будем учитывать положение иона.
\begin{equation*}
	\Psi = \Psi(\vc{R}, \vc{r}_1, \ldots, \vc{r}_n) = \tilde{\Psi}(\vc{R}) \Psi(\vc{r}_1, \ldots, \vc{r}_n).
\end{equation*}
Тут наш гамильтониан:
\begin{equation*}
	\bar{H} = \frac{\bar{p}^2}{2 M} + \frac{m \omega^2 R^2}{2}.
\end{equation*}
И вот гамильтониан диаганализуется с помощью операторов:
\begin{equation*}
	a, a^{\dag} = \sqrt{\frac{\hbar}{2 M \omega}}x \pm i \sqrt{\frac{\hbar \omega}{2M}} \bar{p},
\end{equation*}
тогда гамильтониан представится в простом виде:
\begin{equation*}
	H = \hbar \omega \left(a^\dag a + \frac{1}{2}\right).
\end{equation*}
Классные свойства:
\begin{equation*}
	[a, a^\dag] = 1 = a a^\dag =- a^\dag a;
	\ a \Psi_0 = 0;
\end{equation*}
\begin{equation*}
	H \Psi_0 = \frac{\hbar \omega}{2} \Psi_0;
	\
	\Psi_n = (a^\dag)^n | 0 \rangle.
\end{equation*}
И таким образом через путанный вывод, который, как мне кажется, не смысла приводить здесь получим:
\begin{equation*}
	H \Psi_n = \hbar \omega \left(n + \frac{1}{2}\right) \Psi_n.
\end{equation*}

Вернемся к нашей системе, где для иона важны два электронных уровня и есть собственные колебания вида, как мы выяснили, $E_n = \hbar \omega \left(n+\frac{1}{2}\right)$.
Таким образом наш гамильтониан:
\begin{equation*}
	H = H_0 \vc{E} \cdot \vc{d}.
\end{equation*}
Теперь наши колебания сидят в дипольном моменте:
\begin{equation*}
	\vc{E}(t) = \vc{E}_0 \cos (\Omega t - \vc{k} \cdot \vc{R}),
\end{equation*}
где $\vc{R} = \sqrt{\frac{1}{2 m \hbar \omega}}(a + a^\dag)$.

В электромагнитных волнах мы предполагаем малость амплитуды их колебаний по сравнению с собственными колебаниями ионов.
Короче гамильтониан Лэмбэ-Дикке:
\begin{equation*}
	\bar{H} = H_0 + \frac{E_0}{2}\left(e^{i \Omega t} \left(1 + i k \sqrt{\frac{1}{2 m \hbar \omega}}(a + a^\dag) (\sigma_+ + \sigma_-)\right)\right),
\end{equation*}
где $\Omega = \frac{\varepsilon_1 - \varepsilon_0}{\hbar} \pm \omega_0$. Частота Раби.

\subsubsection*{Обработка квантовой информации на ионах в ловушке}