С прошлого семинара помним \eqref{potential}. А также выражение для величин: $\vc{d}$, $D_{\alpha\beta}$ и $Q = \int \rho(\vc{r}) d^3 \vc{r}$.
Ещё нам понадобиться выражение для октуполя:
\begin{equation*}
	O_{\alpha \beta \gamma} = \int \rho(\vc{r}) (
	15 r_\alpha r_\beta r_\gamma - 3 r_\alpha \delta_{\beta\gamma} - 3 r_\beta \delta_{\alpha \gamma} - 3 r_\gamma \delta_{\alpha\beta} r^2
	) d^3 r.
\end{equation*}
Опустим причины появления такого \textit{чудного} выражения, скажем только что октуполь войдёт в разложение для скалярного потенциала как:
\begin{equation}
	\varphi(\vc{r}) = \int \frac{\rho (\vc{r}')}{|\vc{r} - \vc{r}'|}d^3 r'
	= \frac{Q}{r} + \frac{\vc{d} \vc{r}}{r^3} + \frac{r_\alpha r_\beta D_{\alpha\beta}}{2 r^5} + \frac{O_{\alpha\beta\gamma r_\alpha r_\beta r_\gamma}}{6 r^7} + \ldots
	\label{potential8}
\end{equation}

Вот мы вводим всё новые члены разложения, но давайте же наконец посмотрим на их свойства. Начнём с квадрупольного момента:
\begin{equation*}
	D_{\alpha\beta} = D_{\beta\alpha},
	\hspace{1 cm}
	D_{\alpha\beta} \delta^{\alpha\beta} = \int \rho(\vc{r}) (3 r_\alpha r_\beta \delta^{\alpha\beta} - r^2 \delta_{\alpha\beta} \delta^{\alpha\beta}) d^3 r = 0.
\end{equation*}
Таким образом у квадрупольного момента, с получившимися ограничивающими его свойствами, существует 5 независимых элементов.

Теперь про октупольный момент. У него вообще есть 27 компонент, но сколько же из них независимы. Вариантов когда:
\begin{enumerate}
	\item  все индексы разные $O_{\alpha\beta\gamma}$ всего 6, но из-за индифферентности к перестановкам, они сводятся к 1;
	\item все индексы одинаковые $O_{\alpha\alpha\alpha}$ всега 3;
	\item Две равны и одна отлична: $O_{\alpha\alpha\beta}$ всего 18, но опять из-за перестановок их независимых остается 6.
\end{enumerate}
Итого уже сократили возможное количество независимых переменных к 10.

Теперь свёртками мы можем получить ещё связи и ещё больше сократить число независимых переменных.
\begin{equation*}
	O_{\alpha\beta\gamma}\delta^{\alpha\beta} = \int \rho(\vc{r}) (
	15 r_\alpha r_\beta r_\gamma \delta^{\alpha\beta} - 3 r_\alpha \delta_{\beta\gamma} \delta^{\alpha\beta} - 3 r_\beta \delta_{\alpha \gamma} \delta^{\alpha\beta} - 3 r_\gamma \delta_{\alpha\beta} \delta^{\alpha\beta} r^2
	) d^3 r 
	= \int \rho(\vc{r}) (15 r^2 r_\gamma - 3 r_\gamma r^2 - 3 r_\gamma r^2 - 9 r^2 r_\gamma) d^3 r = 0.
\end{equation*}

Таким образом, по паре каждых компонент свертка -- нуль, то есть ещё 3 условия налагаются на выражение для октупольного момента, оставляя всего \textbf{7} независимых компонент.

\subsubsection*{Задача 11}
Задача аксиально симметрична относительно оси $Oz$, дан потенциал:
\begin{equation*}
	v(r,0) - v_0 \left(1 - \frac{r^2 - a^2}{r \sqrt{r^2 + a^2}}\right), \ r>a.
\end{equation*}
Хочется узнать $v(r,\theta) - ?$ при условии, что $r \gg a$.

Соотвественно раскладываем в ряд, раз $r \gg a$, и получаем:
\begin{equation*}
	v(z,0) = v_0 \left(1 - \left(1 - \frac{a^2}{z^2}\right)\left(1 + \frac{a^2}{z^2}\right)^{-1/2}\right) 
	\simeq
	v_0 \left(1 - \left(1 - \frac{a^2}{z^2}\right)\left(1 + \frac{a^2}{2z^2}\right)\right) 
	=
	v_0 \left(\frac{3 a^2}{2 z^2} - \frac{a^4}{2 z^4}\right).
\end{equation*}

С другой стороны по теории должно было бы получиться разложение \eqref{potential8}. Сравнивая степени в разложении получаем:
\begin{equation*}
	Q = 0, \hspace{1 cm} D_{z z} = 0, \hspace{1 cm} d_z = \frac{3 a^2}{2}v_0,
	\hspace{1 cm} 0_{z z z} = - 3 a^4 v_0.
\end{equation*}

Теперь применим аксиальную симметрию: $\vc{d} = \int \rho(\vc{r}) \vc{r} d^3 \vc{r}$. В дипольном моменте компоненты $d_x = d_y = 0$, что мы получаем так же как в упражнении про усреднение $\rho(x, y) = \rho(-x,-y)$.

Далее $D_{\alpha\alpha} = 0$, значит $D_{xx} + D_{yy} + D_{zz} = 0$ то есть $D_{xx} = - D_{yy}$, но в силу аксиальной симметрии такое возможно лишь если $D_{xx} = D_{yy} = 0$.

Наконец $O_{\alpha\alpha\beta} = 0$. То есть $O_{xxz} + O_{yyz} + O_{zzz} = 0$, тогда получаем: 
\begin{gather*}
	O_{xxz} = O_{yyz} = - \frac{O_{zzz}}{2}	= \frac{3 a^4}{2}v_0\\
	O_{xzx} = O_{yzy} = O_{zxx} = O_{zyy} = \frac{3 a^4}{2}v_0
\end{gather*}

Вариант со всеми разными: $O_{xyz} = \int \rho(x,y,z) (15 x y z) d^3 r = 0$, так как $\rho(x) = \rho(-x)$. И поэтому же $O_{xxx}=O_{yyy}=0$.

И не взятые ещё:
\begin{equation*}
	O_{z zx} = O_{zzy} = O_{xzz} = O_{yzz} = O_{zxz} = O_{zyz} = 0.
\end{equation*}
\begin{equation*}
	O_{xxy} = O_{yyx} = O_{xyx} = O_{yxy} = O_{yxx} = O_{xyy} = 0.
\end{equation*}

Теперь давайте, как нас просят в задаче, подставим $z = r \cos \theta$:
\begin{equation*}
	v_0(r,\theta) = \frac{3 a^2 v_0}{2} \frac{\cos \theta}{r^2} + \left(
	- \frac{a^4 v_0}{2 r^4} \cos^3 \theta + \frac{3 O_{xxz} xxz}{6 r^7} + \frac{3 O_{yyz}yyz}{6 r^7}
	\right) =
	\frac{3 a^2 v_0}{2} \frac{\cos^2 \theta}{r^2} + \frac{\left(\frac{3}{4} \cos\theta \sin^2 \theta - \frac{1}{2} \cos^3 \theta\right)}{r^4}a^4 v_0.
\end{equation*}
/так как по сферической замене: $xxz = r^3 \cos \theta \sin^2 \theta \cos^2 \varphi$, $yyz = r^3 \cos \theta \sin^2 \theta \sin^2 \varphi$/


\subsubsection*{Задача 10}
Нас просят найти диполный момент двух полусфер.
Так как нас спросили только про дипольный момент, а про распределение зарядов не спросили, то мы последнее и не будем находить.
\begin{equation*}
	\varphi(\vc{r}) = \int \frac{\rho(\vc{r}') d^3 \vc{r}'}{|\vc{r} - \vc{r}'|}
	 = \varphi^{(0)} + \varphi^{(1)} + \varphi^{(2)} + \ldots = \sum_{l=0}^\infty \varphi^{(l)}.
\end{equation*}
Известно что (ЛЛII \S 41 его лучше прочитать, потому как мне лень техать выражения для всех величин тут) :
\begin{equation}
	\varphi^{(l)} = \sum_a e_a\sum_{m = -l}^l \sqrt{\frac{4 \pi}{2 l +1}} D_l^{(m)} Y_{l m} (\theta, \varphi) \frac{r_a^l}{R_0^{l+1}}.
	\label{potential_zhest}
\end{equation}
Любой скалярный потенциал мы всегда можем разложить по сферическим гармоникам:
\begin{equation*}
	\varphi(z) = \sum_{l,m} c_{lm} Y_{lm}(\theta,\varphi) R(z).
\end{equation*}
На сфере: $R=z$:
\begin{equation*}
 \varphi(r) =
	\left\{
	\begin{aligned}
		\Phi_0, \ z>0 \\
		-\Phi_0, \ z<0
	\end{aligned}
	\right.
	\hspace{0.5 cm}
	\leadsto
	\hspace{0.5 cm}
	\int \varphi(r) Y_{l m} (\theta, \varphi) = \sin \theta d \theta d\varphi
	=
	i \sqrt{\frac{2 l +1}{4 \pi}} \int_0^{2\pi} \int_0^\pi \cos \theta \varphi(r) \sin \theta d \theta \varphi =
\end{equation*}
\begin{equation*}
	= i \sqrt{\frac{2 l +1}{4 \pi}} \left(
	-\int_0^{\pi/2} \Phi_0 \cos \theta d \cos \theta + \int_{\pi/2}^\pi \Phi_0 \cos \theta d \cos \theta
	\right)
	=
	2 \Phi_0 \pi i \sqrt{\frac{3}{4 \pi}}\left(
	- \frac{\cos ^2 \theta}{2}\bigg|_0^{\pi/2} + \frac{\cos^2 \theta}{2}\bigg|_{\pi/2}^\pi 
	\right).
\end{equation*}
Мы получили, что из \eqref{potential_zhest} взяв как и в выводе формулы до $l=1$:
\begin{equation*}
	2 \pi i \Phi_0 \sqrt{\frac{3}{4\pi}} = \frac{4 \pi}{3} \frac{1}{r} D_l^m = \sqrt{\frac{2 \pi}{3}} i d_z \frac{1}{r}
	\hspace{0.5 cm}
	\Rightarrow
	\hspace{0.5 cm}
	d_z = r \frac{3 \Phi_0}{2}.
\end{equation*}